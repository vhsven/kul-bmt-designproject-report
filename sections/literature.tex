\section{Literature Review}
\subsection{Introduction}
In Belgium one man in three and one woman in four faces cancer before his or
her 75th birthday \cite{kanker}. In 2010 62017 new cases of cancer were
diagnosed in Belgium \cite{kankerliga}. Lung cancer is the second most common
cander for men, and the third for women.
On top of that, it is one of the deadliest cancers
\cite{zheng}. However, an early detection can increase the survival rate up to 70-80\%
\cite{swensen}. Furthermore, research has shown that the detection of lung
cancer in an early stage broadens the amount of treatment options and increases
the amount of invasive surgery\cite{greenlee}.


Due to recent developments in
computed tomography (CT) technology it is now possible to obtain near isotropic, submillimeter resolution images of the complete chest in a single breath hold.
This high resolution has the advantage that it enables visualisation of small
and low-contrast nodules that could hardly be screened in conventional
programs. The downside is that enormous amounts of data are generated
which increases the work load of radiologists, especially since low-dose CT
scans are more and more implemented in routine screenings. Still, this is no
idle measure. Long nodules are very commonly detected on CT scans. Research
shows that up to 51\% of smokers aged 50 years or older have pulmonary lung nodules on CT scans \cite{mahon}.
Therefore, the United States Preventive Services Task Force for example stated that it ``recommends annual screening for lung cancer with low-dose computed tomography (LDCT) in adults aged 55 to 80 years who have a 30 pack-year smoking history and currently smoke or have quit
within the past 15 years'' \cite{ups}.
Therefore, the detection of pulmonary nodules from volumetric computed
tomography (CT) scans if one of the most studied CAD applications
\cite{sluimer}.


Currently, expert radiologists perfrom the investigation of the
CT scans. They use the shape, the texture, the location and the growth rate of
the volume of the nodule as clinical parameter to determine the malignancy of
the nodules and to decide on the diagnosis of lung cancer. A jagged shape
nodule is more likely to be lung cancer than a smoothed one. A fatty, bony,
watery nodule or a mixture of these different contents is less likely to
indicate lung cancer than a nodule that is attached to a vessel. A lung wall
attached nodule is typically diagnosed as benign if the volume-doubling periode
is longer than 400 days \cite{wu}. Nevertheless, the examination of these scans
is a time-consuming task and is not free from errors. Although small nodules are
in principle detectable in CT scans, a non-negligible fraction may be overlooked
if they are situated in a maze of vessels of similar size \cite{ozekes}. Another
problem that arises is the intra- and interreader variability amongst
radiologists in pulmonary lung detection \cite{armato} \cite{hens}. Therefore,
there is a need for a CAD system that can assist the radiologist in the
detection of pulmonary nodules.

\subsection{The biology of lung nodules}
Lung nodules are lung tissue abnormalities that are roughly spherical with a
diameter up to 30 mm. On chest CT scans they appear as a rounded or
irregular opacity. Many types of lung nodules can by distinguished on CT scans.
A centrilobular nodule is separated by several millimeters from the pleural surfaces, fissures and
interlobular septa. They range in size from a few millimeters to 10 millimeters.
A micronodule is less than 3 millimeters in diameter. A ground-glass nodule -or
non-solid nodule- appears on the CT scans as a hazy attenuation in the lung.
This type of nodule does not efface the bronchial and vascular margins. A solid
nodule shows a homogenous soft-tissue attenuation. Finally, a part-solid nodule
exhibits both ground-glass and solid soft-tissue attenuation characteristics
\cite{nodule}. 

The types of nodules stated above can again be categorised. Juxta-vascular
pulmonary nodules have significant connections to their neighbouring vessels.
Pleural tail nodules have only thin connections to the neighbouring pleural
wall. Well-circumscribed nodules on the other hand do not have a connection to
the neighbouring vessels and structures. Juxta-pleural nodules show some degree
of attachment to their neighbouring pleural surface \cite{kostis}.

A number of nodule segmentation algorithms perform well in detecting specific
types of nodules e.g. large, spherical, isolated nodules. However, these CAD
systems show large limitations in detecting e.g. non-isolated nodules that are
connected to the pulmonary wall \cite{keshani}. These algorithms can be usefull
in particular situations, but if a detection algorithms really aims at being an
asset for the radiologist, it should be able to detect all nodules while
refusing as much false positives as possible.






